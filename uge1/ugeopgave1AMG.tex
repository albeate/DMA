\documentclass[a4paper]{article}
\usepackage[utf8]{inputenc} 
\usepackage[danish]{babel} 

\usepackage{amsmath}
\usepackage{amsfonts}
\usepackage{amssymb}
\usepackage{ulem}
\usepackage{verbatim}
\usepackage{graphicx}
\usepackage{xcolor}
\usepackage{marvosym}

\usepackage{listings}
\lstdefinestyle{customc}{
  belowcaptionskip=1\baselineskip,
  breaklines=true,
  frame=L,
  xleftmargin=\parindent,
  language=C,
  showstringspaces=false,
  basicstyle=\footnotesize\ttfamily,
  keywordstyle=\bfseries\color{green!40!black},
  commentstyle=\itshape\color{purple!40!black},
  identifierstyle=\color{blue},
  stringstyle=\color{orange},
}

\lstdefinestyle{customasm}{
  belowcaptionskip=1\baselineskip,
  frame=L,
  xleftmargin=\parindent,
  language=[x86masm]Assembler,
  basicstyle=\footnotesize\ttfamily,
  commentstyle=\itshape\color{purple!40!black},
}

\lstset{escapechar=@,style=customc}

\newcommand*{\mybox}[2]{\colorbox{#1!30}{\parbox{.98\linewidth}{#2}}}
\newcommand{\setR}{\mathbb{R}} 
\newcommand{\setF}{\mathbb{F}} 
\newcommand{\setN}{\mathbb{N}} 
\newcommand{\pder}[2][]{\frac{\partial#1}{\partial#2}}
\newcommand{\RNum}[1]{\uppercase\expandafter{\romannumeral #1\relax}}

\newcommand*{\titleGM}{\begingroup % Create the command for including the title page in the document
	\hbox{ % Horizontal box
		\hspace*{0.2\textwidth} % Whitespace to the left of the title page
		\rule{1pt}{\textheight} % Vertical line
		\hspace*{0.05\textwidth} % Whitespace between the vertical line and title page text
		\parbox[b]{0.75\textwidth}{ % Paragraph box which restricts text to less than the width of the page
			
			{\noindent\Huge\bfseries DMA \\[0.5\baselineskip] }\\[2\baselineskip] % Title
			{\large \textit{Ugeopgave 1}}\\[4\baselineskip] % Tagline or further description
			{\Large \textsc{Beate Berendt Søegaard}}\\ % Author name
			{\Large \textsc{Mathias Larsen}}\\ % Author name
			{\Large \textsc{Simon Rotendahl}} % Author name
			
			\vspace{0.5\textheight} % Whitespace between the title block and the publisher
			{\noindent Datalogi}\\[\baselineskip] \today% Publisher and logo
		}}
		\endgroup}
	
%\title{Navnet på faget  \\ \Large Aflevering}
%\author{Beate B. Søegaard}
\begin{document}
%\maketitle
%\tableofcontents
\titleGM % This command includes the title page

\section*{Del 1}
\lstinputlisting[language=Python]{uge1.py}
\subsubsection*{a}
Den returnerer True, da dels er $x = 17$ og algorithmen når at køre igennem før hi er lig med lo.

\subsubsection*{b}
Den returnerer False, da $x = 14$ ikke er en del af listen.

\subsubsection*{c}
Den returnerer False, da $hi$ ikke er stor nok til at algorithm kan nå, at køre igennem listen før den bliver termineret.

\subsubsection*{d}
Vi for mid til at være hhv.:
\begin{itemize}
\item $4$ TEST SIMON
\item $1$
\item $0$
\end{itemize}

\section*{Del 2}
Tjekker hvorvidt $x$ eksistere i listen, hvis ja så returnerer den True. Den returnerer False dels hvis $x$ ikke eksisterer i listen, men også når $hi$ er for lavt da den ikke når at køre igennem listen. Listen skal også være sorteret ellers giver den et tilfældigt svar. 

\section*{Del 3}
\subsubsection*{a}
Den kan ikke give True på noget som helst tidspunkt, da $x$ skal eksistere i listen da mid er et indekserings tal.

\subsubsection*{b}
Ja, den giver False når $hi$ er for lavt og når et lavt tal har et højere indeks, og vice versa, end mid så vil algorithmen returnerer False.

\section*{Del 4}
Når $n=17$ kan vores implementeret algoritme, skrevet i python, ikke køre da vi vil få en \textit{out of bounce} fejl. I andre tilfælde, såsom C vil du bare få skrald, hvilket kan være hvad som helst. 
\end{document}
