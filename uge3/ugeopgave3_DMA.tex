\documentclass[a4paper]{article}
\usepackage[utf8]{inputenc} 
\usepackage[danish]{babel} 

\usepackage{amsmath}
\usepackage{amsfonts}
\usepackage{amssymb}
\usepackage{ulem}
\usepackage{verbatim}
\usepackage{graphicx}
\usepackage{xcolor}
\usepackage{marvosym}

\usepackage{listings}
\lstdefinestyle{customc}{
  belowcaptionskip=1\baselineskip,
  breaklines=true,
  frame=L,
  xleftmargin=\parindent,
  language=C,
  showstringspaces=false,
  basicstyle=\footnotesize\ttfamily,
  keywordstyle=\bfseries\color{green!40!black},
  commentstyle=\itshape\color{purple!40!black},
  identifierstyle=\color{blue},
  stringstyle=\color{orange},
}

\lstdefinestyle{customasm}{
  belowcaptionskip=1\baselineskip,
  frame=L,
  xleftmargin=\parindent,
  language=[x86masm]Assembler,
  basicstyle=\footnotesize\ttfamily,
  commentstyle=\itshape\color{purple!40!black},
}

\lstset{escapechar=@,style=customc}

\newcommand*{\mybox}[2]{\colorbox{#1!30}{\parbox{.98\linewidth}{#2}}}
\newcommand{\setR}{\mathbb{R}} 
\newcommand{\setF}{\mathbb{F}} 
\newcommand{\setN}{\mathbb{N}} 
\newcommand{\pder}[2][]{\frac{\partial#1}{\partial#2}}
\newcommand{\RNum}[1]{\uppercase\expandafter{\romannumeral #1\relax}}

\newcommand*{\titleGM}{\begingroup % Create the command for including the title page in the document
	\hbox{ % Horizontal box
		\hspace*{0.2\textwidth} % Whitespace to the left of the title page
		\rule{1pt}{\textheight} % Vertical line
		\hspace*{0.05\textwidth} % Whitespace between the vertical line and title page text
		\parbox[b]{0.75\textwidth}{ % Paragraph box which restricts text to less than the width of the page
			
			{\noindent\Huge\bfseries DMA \\[0.5\baselineskip] }\\[2\baselineskip] % Title
			{\large \textit{Ugeopgave 1}}\\[4\baselineskip] % Tagline or further description
			{\Large \textsc{Beate Berendt Søegaard}}\\ % Author name
			{\Large \textsc{Mathias Larsen}}\\ % Author name
			{\Large \textsc{Simon Rotendahl}} % Author name
			
			\vspace{0.5\textheight} % Whitespace between the title block and the publisher
			{\noindent Datalogi}\\[\baselineskip] \today% Publisher and logo
		}}
		\endgroup}
	
%\title{faget  \\ \Large Aflevering #}
%\author{navn}
\begin{document}
%\maketitle
\titleGM % This command includes the title page
%\tableofcontents

\section*{Del 1}
I den første sektion ser vi på størrelsesorden af de følgende følger ved brug af formelsamliingen, $O1-O8$ og $S1-S8$, for DMA.\\

$n + log_2 n$:
\begin{align*}
	(1: S2): log_2 n \in O(n)\\
	(2): n \in \Theta(n) \Rightarrow n \in \mathcal{O}(n)\\
	(3: 1, 2, O8):  n + log_2 n \in \mathcal{O}(n)\\
	(4: S3): n \in o(n^2)\\
	(5: 4, 3):  n + log_2 n \notin \Theta(n^2)
\end{align*}
$n^2 + 2^n$:
\begin{align*}
	(1: S5): n^2 \in o(2^n)\\
	(2: S8): n^2 + 2^n \in \Theta(2^n)\\
	(3: 2): n^2 + 2^n \notin \Theta(n^2)
\end{align*}
$n^2 + n log_{10}n$:
\begin{align*}
	(1: S2): log_{10}n \in \mathcal{O}(n)\\
	(2: S7, 1): n log_{10}n \in o(n*n) = o(n^2)\\
	(3): n^2 \in \Theta(n^2)\\
	(4: S8, 2, 3): n^2 + n log_{10} n \in \Theta(n^2)
\end{align*}
$(n+\sqrt{n})^2 = n^2 + n + 2n\sqrt{n}$:
\begin{align*}
	(1: S6): 2n\sqrt{n} \in \Theta(n\sqrt{n})\\
	(2: S7, S8): n + 2n\sqrt{n} \in \Theta(n\sqrt{n})\\
	(3: S7, S8, 2): n^2 + n + 2n\sqrt{n} \in \Theta(n^2)
\end{align*}
$n^2(3 + sqrt{n}) = 3n^2 + n^2n^{\frac{1}{2}}$:
\begin{align*}
	(1: S6): 3n^2 \in \Theta(n^2)\\
	(2: S8, 1): 3n^2 + n^2n^{\frac{1}{2}} \in \Theta(n^2n^{\frac{1}{2}}\\
		(3: 2): n^2(3 + \sqrt{n}) \notin \Theta(n^2)
\end{align*}

\section*{Del 2}
\subsection*{a} % (fold)
\label{sub:a}
Vi vil beregne de tre første værdier af følgerne.\\

% subsection  (end)

\subsection*{b} % (fold)
\label{sub:b}
Følgerne sorteret hhv. til størrelseorden.\\
Vi fra forrige delopgave at 
\begin{align*}
a_n &= 10 \\
b_n &= \frac{2n^3+3n^2+n}{6} \\
c_n &= \frac{n^2}{10} \\
d_n &= \Big(\frac{3}{2}\Big)^n
\end{align*}
Jf. regnereglerne $S6$ og $S3$ kan vi fjerne konstanterne og forkorte vores følger således,
\begin{align*}
a_n &= \mathcal{O}(1)\\
b_n &= \mathcal{O}(2n^3+3n^2+n^1) = \mathcal{O}(2n^3)\\
c_n &= \mathcal{O}(n^2) \\
d_n &= \mathcal{O}\Big(\Big(\frac{3}{2}\Big)^n\Big)
\end{align*}

Det ses let at, følgen $a_n$ er den mindste blandt følgerne, da den er lig nul. Jf. regl $S5$ er $b_n = \mathcal{O}(b_n)$. Jf. regl $S3$ ser vi at $c_n = \mathcal{O}(b_n)$ og slutvis, jf. regl $O1$, har vi at $c_n = \mathcal{O}(d_n)$. Altså er rækkefølgen
\begin{align*}
&a_n \\
&c_n \\
&b_n \\
&d_n 
\end{align*}
% subsection  (end)


\section*{Del 3}
Vi får givet den følgende sumfølge
\begin{align*}
\sum_{k=0}^{n} (2k+1) \\
\end{align*}
Vi kan splitte vores sumfølge op og får 
\begin{align*}
2\sum_{k=0}^{n}k + \sum_{k=0}^{n} 1 &=\\ 
\end{align*}
Jf. formelsamlingen for DMA har vi, at
\begin{align*}
&= 2\frac{n(n+1)}{2}+n \\
\end{align*}
Men da vi starter fra $k=0$ omskrives formlen til 
\begin{align*}
&= 2\frac{n(n+1)}{2}+n+1 \\
&= (n+1)n+n+1 \\
&= (n+1)^2
\end{align*}
Således har vi det følgende eksplicit udtryk for sumfølgen
\begin{align*}
\sum_{k=0}^{n} (2k+1) = (n+1)^2
\end{align*}
\end{document}