\documentclass[a4paper]{article}
\usepackage[utf8]{inputenc} 
\usepackage[danish]{babel} 

\usepackage{amsmath}
\usepackage{amsfonts}
\usepackage{amssymb}
\usepackage{ulem}
\usepackage{verbatim}
\usepackage{graphicx}
\usepackage{xcolor}
\usepackage{marvosym}

\usepackage{tikz}
\usetikzlibrary{calc,shapes.multipart,chains,arrows}
\tikzset{
squarecross/.style={
    draw, rectangle,minimum size=18pt, fill=orange!80,
    inner sep=0pt, text=black,
    path picture = {
        \draw[black]
        (path picture bounding box.north west) --
        (path picture bounding box.south east)
        (path picture bounding box.south west) --
        (path picture bounding box.north east);
    }
  }
}

\usepackage{listings}
\lstdefinestyle{customc}{
  belowcaptionskip=1\baselineskip,
  breaklines=true,
  frame=L,
  xleftmargin=\parindent,
  language=C,
  showstringspaces=false,
  basicstyle=\footnotesize\ttfamily,
  keywordstyle=\bfseries\color{green!40!black},
  commentstyle=\itshape\color{purple!40!black},
  identifierstyle=\color{blue},
  stringstyle=\color{orange},
}

\lstdefinestyle{customasm}{
  belowcaptionskip=1\baselineskip,
  frame=L,
  xleftmargin=\parindent,
  language=[x86masm]Assembler,
  basicstyle=\footnotesize\ttfamily,
  commentstyle=\itshape\color{purple!40!black},
}

\lstset{escapechar=@,style=customc}

\newcommand*{\mybox}[2]{\colorbox{#1!30}{\parbox{.98\linewidth}{#2}}}
\newcommand{\setR}{\mathbb{R}} 
\newcommand{\setF}{\mathbb{F}} 
\newcommand{\setN}{\mathbb{N}} 
\newcommand{\setZ}{\mathbb{Z}}
\newcommand{\pder}[2][]{\frac{\partial#1}{\partial#2}}
\newcommand{\RNum}[1]{\uppercase\expandafter{\romannumeral #1\relax}}

\newcommand*{\titleGM}{\begingroup % Create the command for including the title page in the document
	\hbox{ % Horizontal box
		\hspace*{0.2\textwidth} % Whitespace to the left of the title page
		\rule{1pt}{\textheight} % Vertical line
		\hspace*{0.05\textwidth} % Whitespace between the vertical line and title page text
		\parbox[b]{0.75\textwidth}{ % Paragraph box which restricts text to less than the width of the page
			
			{\noindent\Huge\bfseries DMA \\[0.5\baselineskip] }\\[2\baselineskip] % Title
			{\large \textit{Ugeopgave 5}}\\[4\baselineskip] % Tagline or further description
			{\Large \textsc{Beate Berendt Søegaard}}\\ % Author name
			{\Large \textsc{Mathias Larsen}}\\ % Author name
			{\Large \textsc{Simon Rotendahl}} % Author name
			
			\vspace{0.5\textheight} % Whitespace between the title block and the publisher
			{\noindent Datalogi}\\[\baselineskip] \today% Publisher and logo
		}}
		\endgroup}
	
%\title{faget  \\ \Large Aflevering #}
%\author{navn}
\begin{document}
%\maketitle
\titleGM % This command includes the title page
%\tableofcontents
\section*{Del 1}
\subsection*{1}
Vi får givet en table med GCD, \textit{Greatest Common Divisor}

\begin{figure} [h]
\begin{center}
\begin{tabular}{|c||c|c|c|c|c|c|c|c|c|c|c|c|c|c|c|}\hline
&1&2&3&4&5&6&7&8&9&10&11&12&13&14&15\\\hline\hline
1&1&1&1&1&1&1&1&1&1&1&1&1&1&1&1\\\hline
2&&1&2&1&2&1&2&1&2&1&2&1&2&1&2\\\hline
3&&&1&2&3&1&2&3&1&2&3&1&2&3&1\\\hline
4&&&&1&2&2&3&1&2&2&3&1&2&2&3\\\hline
5&&&&&1&2&3&&3&1&2&3&4&3&1\\\hline
6&&&&&&1&2&2&2&3&3&1&2&2&2\\\hline
7&&&&&&&1&2&3&3&4&4&3&1&2\\\hline
8&&&&&&&&1&2&2&4&2& &3&3\\\hline
9&&&&&&&&&1&2&3&2&3&4&3\\\hline
10&&&&&&&&&&1&2&2&3&3&2\\\hline
11&&&&&&&&&&&1&2&3&4&4\\\hline
12&&&&&&&&&&&&1&2&2&2\\\hline
13&&&&&&&&&&&&&1&2&3\\\hline
14&&&&&&&&&&&&&&1&2\\\hline
15&&&&&&&&&&&&&&&1\\\hline
\end{tabular}
\end{center}
%\caption{Antal trin i beregningen af $\GCD(a,b)$}\label{antaltrin}

\end{figure}

\begin{lstlisting}
(1) fun gcd(a,b)
(2)   if b = then 
(3)      return a
(4)   else: 
(5)      return gcd(b, a mod b)
\end{lstlisting}
\begin{align*}
\text{gcd}(8,5) = 1
\end{align*}
\begin{align*}
\text{gcd}(13,8) = 1
\end{align*}
Dvs. de manglende tal er hhv. 1 og 1. 
 
\subsection*{2}
Lad $t_n$ være det højeste antal trin der skal benyttes til at bestemme GCD(a,b) når $n \geq a \geq b > 0$. Ved at benytte ovenstående figur har vi at $t_1,\cdots,t_{15}$ er,
\begin{align*}
t_1 = 1 \\
t_2 = 1 \\
t_3 = 2 \\
t_4 = 2 \\
t_5 = 3 \\
t_6 = 2 \\
t_7 = 3 \\
t_8 = 3 \\
t_9 = 3 \\
t_{10} = 3 \\
t_{11} = 4 \\
t_{12} = 4 \\
t_{13} = 4 \\
t_{14} = 4 \\
t_{15} = 4
\end{align*}

\subsection*{3}
Vi kan vise at $t_n$ er $\mathcal{O}(n)$. Vi ved at det taget $n$-tid at køre samligningslinjerne i kode igennem samt at vores returneringslinjer tager konstant tid at køre igennem, derfor får vi at
\begin{align*}
t_n = \mathcal{O}(n) \cdot \mathcal{O}(1) \cdot \mathcal{O}(1) = \mathcal{O}(n)
\end{align*}

\subsection*{4}
Ud fra vores pseudokode og konklusionen vedrørende køretid fra delopgave 1-3, \\
kan $t_n$ ikke være $\mathcal{O}(1)$.

\subsection*{5}
Nej, ud fra figur 2, fra opgaveformuleringen, er $t_n = \Theta(n^2)$.

\section*{Del 2}  
\subsection*{1}
Vi får givet at 
\begin{align*}
P(n) : 5\;|\;(6^n-5n+4) \quad \text{for} \; n \in \setZ
\end{align*}
Dvs. at højreside divideret med 5 skal være lig med et heltal $I \in \setZ$. Vi omskriver udtrykket $P(n)=I$. \\
Først sætter vi $(6^n-5n+4)=x$ så får vi at $5\;|\;x=I$, som vi derefter kan omskrive til $x=5(I)$. Da dette udtryk er nemmere, at arbejde med end $(6^n-5n+4)$, kan vi antage at svaret til $(6^n-5n+4)$ kan løses af et multuplum af 5 og et positivt heltal $I$.\\
Altså får vi udtrykket
\begin{align*}
P(n) & : 5\;|\;(6^n-5n+4) = I \quad I \in \setZ^+, \; n \in \setZ^+
\end{align*}
\begin{center}
Eller
\end{center}
\begin{align*}
P(n) & : (6^n-5n+4) = 5(I) \quad I \in \setZ^+, \; n \in \setZ^+ 
\end{align*}

\subsection*{2}
\begin{align*}
P(n) & : (6^n-5n+4) = 5(I) \\
P(1) & : (6^1-5(1)+4) = 5(I) \Rightarrow 6-5+4 = 5(I) = 5(1)\\
P(2) & : (6^2-5(2)+4) = 5(I) \Rightarrow 36-10+4 = 5(I) = 5(6)\\
P(3) & : (6^3-5(3)+4) = 5(I) \Rightarrow 216-15+4 = 5(I) = 5(41)\\
P(4) & : (6^4-5(4)+4) = 5(I) \Rightarrow 1298-20+4 = 5(I) = 5(256)\\
P(5) & : (6^5-5(5)+4) = 5(I) \Rightarrow 7776-25+4 = 5(I) = 5(1551)
\end{align*}

\subsection*{3}
Vi har følgen for $b_n$,
\begin{align*}
b_n = 6^n-5n+4
\end{align*}
og følgen for $b_{n+1}$ vil se ud på følgende måde for $n\in\setZ^+$, altså

\begin{align*}
b_{n+1} : 6^{k+1}-5(k+1)+4 &= 5(I_k) \\
6^{k+1}-5k -5 +4 &= 5(I_k) \\
6\cdot6^k-5k+4-5 &= 5(I_k) \\
(5+1)\cdot6^k-5k+4-5 &= 5(I_k) \\
6^k-5k+4+5\cdot6^k-5 &= 5(I_k) 
\end{align*}
Da $b_n = 6^n-5n+4$ kan vi omskrive ovenstående på følgende måde
\begin{align*}
b_{n+1} : b_n + 5\cdot6^k-5 &= 5(I)
\end{align*}

$5(I)$ og $5(I_k)$ skal forstås således at de begge er $\setZ^+$, men ikke nødvendigvis ens. 

\subsection*{4}
Vi har vist i delopgave 2, at $P(n)$ er sand for $n=1\cdots5$.
\begin{align*}
P(k+1) : 6^k-5k+4+5\cdot6^k-5 &= 5(I) \\
\end{align*}
Da $P(k) : 6^n-5n+4$ som er lig med $b_n$ som også er lige med $5(I)$, dvs.
\begin{align*}
5(I)+5\cdot6^k-5 &= 5(I)
\end{align*}
Vi dividere med 5 over hele ligningen, og får
\begin{align*}
I+6^k-1 = I_k
\end{align*}
Da både $I, k, I_k \in \setZ^+$ ses det let at dette udtryk er sand. 

\subsection*{5}
Vi har i tidligere delopgaver vist vha. induktion at $\forall n>0\;P(n)$, hvor 
\begin{align*}
P(n):5\;|\;(6^n-5n+4)
\end{align*}
I delopgave 2-2 udførte vi 5 basistrin for $n=1\cdots5$. Derefter begyndte vi vores første induktiontrin i delopgave 2-3. Her sammenknyttede vi $b_n$ og $b_{n+1}$ til at være $b_{n+1} = b_n + 5\cdot6^n-5$. I delopgave 2-4 færdiggjorde vi vores udregninger fra delopgave 3-2 samt redegjorde for at $P(k+1)$ er sandt, kombineret med vores basistrin som var sandt, kan vi konkludere at $P(n)$ er sand.
\end{document}