\documentclass[a4paper]{article}
\usepackage[utf8]{inputenc} 
\usepackage[danish]{babel} 

\usepackage{amsmath}
\usepackage{amsfonts}
\usepackage{amssymb}
\usepackage{ulem}
\usepackage{verbatim}
\usepackage{graphicx}
\usepackage{xcolor}
\usepackage{marvosym}

\usepackage[margin=.7in]{geometry}

\usepackage{listings}
\lstdefinestyle{customc}{
  belowcaptionskip=1\baselineskip,
  breaklines=true,
  frame=L,
  xleftmargin=\parindent,
  language=C,
  showstringspaces=false,
  basicstyle=\footnotesize\ttfamily,
  keywordstyle=\bfseries\color{green!40!black},
  commentstyle=\itshape\color{purple!40!black},
  identifierstyle=\color{blue},
  stringstyle=\color{orange},
}

\lstdefinestyle{customasm}{
  belowcaptionskip=1\baselineskip,
  frame=L,
  xleftmargin=\parindent,
  language=[x86masm]Assembler,
  basicstyle=\footnotesize\ttfamily,
  commentstyle=\itshape\color{purple!40!black},
}

\lstset{escapechar=@,style=customc}

\newcommand*{\mybox}[2]{\colorbox{#1!30}{\parbox{.98\linewidth}{#2}}}
\newcommand{\setR}{\mathbb{R}} 
\newcommand{\setF}{\mathbb{F}} 
\newcommand{\setN}{\mathbb{N}} 
\newcommand{\pder}[2][]{\frac{\partial#1}{\partial#2}}
\newcommand{\RNum}[1]{\uppercase\expandafter{\romannumeral #1\relax}}

\newcommand*{\titleGM}{\begingroup % Create the command for including the title page in the document
	\hbox{ % Horizontal box
		\hspace*{0.2\textwidth} % Whitespace to the left of the title page
		\rule{1pt}{\textheight} % Vertical line
		\hspace*{0.05\textwidth} % Whitespace between the vertical line and title page text
		\parbox[b]{0.75\textwidth}{ % Paragraph box which restricts text to less than the width of the page
			
			{\noindent\Huge\bfseries DMA \\[0.5\baselineskip] }\\[2\baselineskip] % Title
			{\large \textit{Ugeopgave 2}}\\[4\baselineskip] % Tagline or further description
			{\Large \textsc{Beate Berendt Søegaard}}\\ % Author name
			%{\Large \textsc{Mathias Larsen}}\\ % Author name
			%{\Large \textsc{Simon Rotendahl}} % Author name
			
			\vspace{0.5\textheight} % Whitespace between the title block and the publisher
			{\noindent Datalogi}\\[\baselineskip] \today% Publisher and logo
		}}
		\endgroup}
	
%\title{faget  \\ \Large Aflevering #}
%\author{navn}
\begin{document}
%\maketitle
%\tableofcontents
\titleGM % This command includes the title page

\section*{Del 1}
Der er 3 inversion i listen [2,1,8,4,3,6], da kriterier for algorithm er således,
\begin{align}
(i,j) = A[i] > A[j] \wedge i < j
\end{align}
Hvor $A[i]$ og $A[j]$ er tallet på det vilkårlige indeksplads og $i$ og $j$ er selve indekset. \\

De tre inversion er altså 
\begin{align*}
(i,j) = (2,1) \rightarrow A[2] > A[1] \wedge 0 < 1 \\
(i,j) = (8,4) \rightarrow A[8] > A[4] \wedge 2 < 3 \\
(i,j) = (4,3) \rightarrow A[4] > A[3] \wedge 3 < 4 
\end{align*}


\section*{Del 2}
Jf. CLRS A.1 har vi den aritmetiske række 
\begin{align*}
\sum_{k=1}^{n} k = 1 + 2 + \cdots + n = \sum_{k=1}^{n} k = \frac{n(n+1)}{2}
\end{align*}

Ved at tage en vikårlig liste, $A1 = [2,4,6,8,10]$ og køre den igennem trin for trin, som nedenfor,
\begin{align*}
(2,4), \quad\quad n-1 \\
(4,6), \quad\quad n-2 \\
(6,8), \quad\quad n-3 \\
\vdots \quad\quad\quad\quad \vdots \quad\\
(n-1,n),\quad1\quad 	
\end{align*}
%må finde på noget mere effektiv
Får man 
\begin{align*}
(n-1)+(n-2)+(n-3)+\cdots+1
\end{align*}
Så har man, at den aritmetiske række kan omskrives til
\begin{align*}
\sum_{k=1}^{n-1} k = \frac{n(n-1)}{2}
\end{align*}
Som er det maksimale antal af inversion $A$ kan have. 


\section*{Del 3}
\begin{lstlisting}
(1) A = [2,8,4,3,5]
(2) inversion(A)
(3)	 n = A.lenght
(4)	 m = 0
(5)	   for i=0 to n-1:
(6)	     for j=0 to n-1:
(7)		   if A[i] > A[j] and i < j then:
(8)			 m += 1 
(9)	 return m
\end{lstlisting}

\section*{Del 4}
(3), (5), (7) koster $c(n)$ og (4), (9) koster $c(1)$. Det der koster mest er løkke nummer to, (6), der koster $c(n^2)$. Derfor er køretiden $T(n) = \Theta(n^2)$.
\end{document}